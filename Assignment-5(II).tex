
\documentclass[journal,12pt,twocolumn]{IEEEtran}
%
\usepackage{setspace}
\usepackage{gensymb}
\singlespacing
\usepackage[cmex10]{amsmath}
\usepackage{siunitx}
\usepackage{amsthm}

\usepackage{mathrsfs}

\usepackage{txfonts}
\usepackage{stfloats}

\usepackage{steinmetz}
\usepackage{cite}
\usepackage{cases}
\usepackage{subfig}
\usepackage{longtable}
\usepackage{multirow}
\usepackage{enumitem}
\usepackage{mathtools}
\usepackage{tikz}
\usepackage{circuitikz}
\usepackage{verbatim}
\usepackage{tfrupee}
\usepackage[breaklinks=true]{hyperref}
\usepackage{tkz-euclide} % loads  TikZ and tkz-base
\usetikzlibrary{calc,math}
\usetikzlibrary{fadings}
\usepackage{listings}
    \usepackage{color}                                            %%
    \usepackage{array}                                            %%
    \usepackage{longtable}                                        %%
    \usepackage{calc}                                             %%
    \usepackage{multirow}                                         %%
    \usepackage{hhline}                                           %%
    \usepackage{ifthen}                                           %%
  %optionally (for landscape tables embedded in another document): %%
    \usepackage{lscape}     
\usepackage{multicol}
\usepackage{chngcntr}
\DeclareMathOperator*{\Res}{Res}

\renewcommand\thesection{\arabic{section}}
\renewcommand\thesubsection{\thesection.\arabic{subsection}}
\renewcommand\thesubsubsection{\thesubsection.\arabic{subsubsection}}

\renewcommand\thesectiondis{\arabic{section}}
\renewcommand\thesubsectiondis{\thesectiondis.\arabic{subsection}}
\renewcommand\thesubsubsectiondis{\thesubsectiondis.\arabic{subsubsection}}

\hyphenation{op-tical net-works semi-conduc-tor}
\def\inputGnumericTable{}                                 %%

\lstset{
%language=C,
frame=single, 
breaklines=true,
columns=fullflexible
}
\begin{document}
%


\newtheorem{theorem}{Theorem}[section]
\newtheorem{problem}{Problem}
\newtheorem{proposition}{Proposition}[section]
\newtheorem{lemma}{Lemma}[section]
\newtheorem{corollary}[theorem]{Corollary}
\newtheorem{example}{Example}[section]
\newtheorem{definition}[problem]{Definition}
\newcommand{\BEQA}{\begin{eqnarray}}
\newcommand{\EEQA}{\end{eqnarray}}
\newcommand{\define}{\stackrel{\triangle}{=}}
\bibliographystyle{IEEEtran}
\providecommand{\mbf}{\mathbf}
\providecommand{\pr}[1]{\ensuremath{\Pr\left(#1\right)}}
\providecommand{\qfunc}[1]{\ensuremath{Q\left(#1\right)}}
\providecommand{\sbrak}[1]{\ensuremath{{}\left[#1\right]}}
\providecommand{\lsbrak}[1]{\ensuremath{{}\left[#1\right.}}
\providecommand{\rsbrak}[1]{\ensuremath{{}\left.#1\right]}}
\providecommand{\brak}[1]{\ensuremath{\left(#1\right)}}
\providecommand{\lbrak}[1]{\ensuremath{\left(#1\right.}}
\providecommand{\rbrak}[1]{\ensuremath{\left.#1\right)}}
\providecommand{\cbrak}[1]{\ensuremath{\left\{#1\right\}}}
\providecommand{\lcbrak}[1]{\ensuremath{\left\{#1\right.}}
\providecommand{\rcbrak}[1]{\ensuremath{\left.#1\right\}}}
\theoremstyle{remark}
\newtheorem{rem}{Remark}
\newcommand{\sgn}{\mathop{\mathrm{sgn}}}
\providecommand{\abs}[1]{\left\vert#1\right\vert}
\providecommand{\abs}[1]{\lvert#1\rvert} 
\providecommand{\res}[1]{\Res\displaylimits_{#1}} 
\providecommand{\norm}[1]{\left\lVert#1\right\rVert}
%\providecommand{\norm}[1]{\lVert#1\rVert}
\providecommand{\mtx}[1]{\mathbf{#1}}
\providecommand{\mean}[1]{E\left[ #1 \right]}
\providecommand{\fourier}{\overset{\mathcal{F}}{ \rightleftharpoons}}
%\providecommand{\hilbert}{\overset{\mathcal{H}}{ \rightleftharpoons}}
\providecommand{\system}{\overset{\mathcal{H}}{ \longleftrightarrow}}
	%\newcommand{\solution}[2]{\textbf{Solution:}{#1}}
\newcommand{\solution}{\noindent \textbf{Solution: }}
\newcommand{\cosec}{\,\text{cosec}\,}
\providecommand{\dec}[2]{\ensuremath{\overset{#1}{\underset{#2}{\gtrless}}}}
\newcommand{\myvec}[1]{\ensuremath{\begin{pmatrix}#1\end{pmatrix}}}
\newcommand{\mydet}[1]{\ensuremath{\begin{vmatrix}#1\end{vmatrix}}}
\numberwithin{equation}{subsection}
\makeatletter
\@addtoreset{figure}{problem}
\makeatother
\let\StandardTheFigure\thefigure
\let\vec\mathbf
\renewcommand{\thefigure}{\theproblem}
\def\putbox#1#2#3{\makebox[0in][l]{\makebox[#1][l]{}\raisebox{\baselineskip}[0in][0in]{\raisebox{#2}[0in][0in]{#3}}}}
     \def\rightbox#1{\makebox[0in][r]{#1}}
     \def\centbox#1{\makebox[0in]{#1}}
     \def\topbox#1{\raisebox{-\baselineskip}[0in][0in]{#1}}
     \def\midbox#1{\raisebox{-0.5\baselineskip}[0in][0in]{#1}}
\vspace{3cm}
\title{ASSIGNMENT-5}
\author{R.YAMINI}
\maketitle
\newpage
\bigskip
\renewcommand{\thefigure}{\theenumi}
\renewcommand{\thetable}{\theenumi}
%
\section{QUESTION No-2.98 (Quadratic forms)}
\item Find the area lying above x-axis and included between the circle $\vec{x}^T\vec{x}-8\myvec{1 & 0}\vec{x}=0$  and inside of the parabola $y^2 = 4x$.
%

%
\section{Solution}
Given equation of the circle 
\begin{align}
 \vec{x}^T\vec{x}-8\myvec{1 & 0}\vec{x}=0.\label{eq:eqn1}  
\end{align}
We know that the general equation of a circle is given by
\begin{align}
  \vec{x}^T\vec{x}+2\vec{u}^T\vec{x}+f=0  
\end{align}
We have $\vec{u} = \myvec{-4 \\ 0}$ and $f = 0$.Thus we have the center and radius as 
\begin{align}
\vec{c} = -\vec{u} = \myvec{4 \\ 0}
\end{align} and 
\begin{align}
r= \sqrt{\vec{u^T}\vec{u} - f} = 4
\end{align} respectively.
Given equation of the parabola 
\begin{align}
 \vec{x}^T\myvec{0&0 \\ 0&1}\vec{x}-4\myvec{1&0}\vec{x}=0 \label{eq:eqn2}  
\end{align}
The plot of the above two curves is
\numberwithin{figure}{section}
\begin{figure}[ht]
\centering
\includegraphics[width=\columnwidth]{Area.PNG}
\caption{Plot of the curves}
\label{Plot of the curves}
\end{figure}
\\
We can see that y-axis is the tangent for both the circle and the parabola.
Now the parametric equation of the tangent is given by
\begin{align}
    \vec{x}= \myvec{0 \\ 0} +\lambda\myvec{0 \\ 1} 
    \\
    = \myvec{0 \\ \lambda} \label{eq:eqn3}
\end{align}
Now substitute \eqref{eq:eqn3} in \eqref{eq:eqn1},
\begin{align}
\myvec{0&\lambda}\myvec{0 \\ \lambda}-8\myvec{1&0}\myvec{0 \\ \lambda}=0
\\
\implies \lambda=0
\end{align}
Thus the point of contact of the tangent with the circle is $A=\myvec{0 \\ 0}$.
Now to find the point of contact of the tangent with the parabola,substiute \eqref{eq:eqn3} in \eqref{eq:eqn2}
\begin{align}
    \myvec{0&\lambda}\myvec{0&0 \\ 0&1}\myvec{0 \\ \lambda}-4\myvec{1&0}\myvec{0 \\ \lambda}=0
    \\
    \implies \lambda=0
\end{align}
Thus the point of contact of the parabola with the tangent is $\vec{A}=\myvec{0 \\ 0}$.
Thus $\vec{A}=\myvec{0 \\ 0}$ is the point of contact for the circle,the parabola and the tangent.
Now to find the point of contact of the circle and parabola with the chord.
The chord equation in parametric form is given as
\begin{align}
    \vec{x}= \myvec{4 \\ 0} +\lambda\myvec{0 \\ 1} 
    \\
    = \myvec{4 \\ \lambda} \label{eq:eqna}
\end{align}
Now substitute \eqref{eq:eqna} in \eqref{eq:eqn1} and \eqref{eq:eqn2}
\begin{align}
\myvec{4&\lambda}\myvec{4 \\ \lambda}-8\myvec{1&0}\myvec{4 \\ \lambda}=0
\\
\implies \lambda=\pm 4
\end{align}
Since the area to be found is above the x-axis we consider $\lambda = +4$ and substituting in \eqref{eq:eqna} we obtain the point of contact to be
$\vec{C}=\myvec{4 \\ 4}$.
Now to find the point of contact of the chord with parabola,sustitute \eqref{eq:eqna} in \eqref{eq:eqn2},
\begin{align}
    \myvec{4&\lambda}\myvec{0&0 \\ 0&1}\myvec{4 \\ \lambda}-4\myvec{1&0}\myvec{4 \\ \lambda}=0
    \\
    \implies \lambda=\pm 4
\end{align}
Since the area to be found is above the x-axis we consider $\lambda = +4$ and substituting in \eqref{eq:eqna} we obtain the point of contact to be
$\vec{C}=\myvec{4 \\ 4}$.
Thus $\vec{C}=\myvec{4 \\ 4}$ is the point of contact for the circle,the parabola and the chord.
\\
\brak{\textbf{OR}}
Now to find the point of contact of the circle and parabola with the chord.
Since the chord is parallel to the tangent their equation differ by a constant.
So, let the equation of the chord be of the form 
\begin{align}
  \vec{x}=\myvec{0 \\ 0}+\brak{\lambda +c}\myvec{0 \\ 1}  
  \\
  = \myvec{0 \\ \brak{\lambda + c}}\label{eq:eqn4}
\end{align}
Now substitute \eqref{eq:eqn4} in \eqref{eq:eqn2} we have,
\\
\\
Now to find the area bounded above the x-axis,the parabola and the circle.
From fig.2.1 the area to be calculated is $AOBCA$.
\begin{align}
Ar\brak{AOBCA} = Ar\brak{ACOA} + Ar\brak{OCBO}
\\
= A_{1} + A_{2}
\end{align}
To calculate $A_{1}$:
$A_{1}$ is the area enclosed by the parabola $y^2=4x$ and the line $OC$.Thus
\begin{align}
    A_{1} = \frac{2}{3}\brak{AO}\brak{OC}
    \\
    = \frac{2}{3}\brak{4}\brak{4} = \frac{32}{3}
\end{align}
To calculate $A_{2}$: $A_{2}$ is one fourth of the area of the circle.
\begin{align}
    A_{2} = \frac{1}{4}\brak{\pi r^2}
    \\
    = \frac{1}{4}\brak{16\pi}
    \\
    = 4\pi
\end{align}
Now substituting (2.0.5) and (2.0.8) in (2.0.3) we get
\begin{align}
    A_{1} + A_{2} = \frac{32}{3} + 4\pi
    \\
    = 4\brak{\frac{8}{3} + \pi} 
\end{align}
Thus (2.0.l0) is the required area.

%%%%%%%%%%%%%%%%%%%%%%%%%%%%%%%%%%%%%%%%%%%%%%%%%%%%%%%%%%%%%%%%%

\section{Solution}
Given equation of the circle 
\begin{align}
 \vec{x}^T\vec{x}-8\myvec{1 & 0}\vec{x}=0.\label{eq:eqn1}  
\end{align}
We know that the general equation of a circle is given by
\begin{align}
  \vec{x}^T\vec{x}+2\vec{u}^T\vec{x}+f=0  
\end{align}
We have $\vec{u} = \myvec{-4 \\ 0}$ and $f = 0$.Thus we have the center and radius as 
\begin{align}
\vec{c} = -\vec{u} = \myvec{4 \\ 0}
\end{align} and 
\begin{align}
r= \sqrt{\vec{u^T}\vec{u} - f} = 4
\end{align} respectively.
Given equation of the parabola 
\begin{align}
 \vec{x}^T\myvec{0&0 \\ 0&1}\vec{x}-4\myvec{1&0}\vec{x}=0 \label{eq:eqn2}  
\end{align}
The plot of the above two curves is
\numberwithin{figure}{section}
\begin{figure}[ht]
\centering
\includegraphics[width=\columnwidth]{Area.PNG}
\caption{Plot of the curves}
\label{Plot of the curves}
\end{figure}
\\
We can see that y-axis is the tangent for both the circle and the parabola.
Let the parametric equation of the tangent be
\begin{align}
    \vec{x} = \vec{q}+\lambda\vec{m} \label{eq:eqnb}
\end{align}
Substitute \eqref{eq:eqnb} in \eqref{eq:eqn1} 
\begin{align}
  \brak{\vec{q}+\lambda\vec{m}}^{\top}\brak{\vec{q}+\lambda\vec{m}}-8\myvec{1&0}\brak{\vec{q}+\lambda\vec{m}} =0 \label{eq:eqnI}
 \end{align}
Now substitute \eqref{eq:eqnb} in \eqref{eq:eqn2} 
\begin{align}
    \brak{\vec{q}+\lambda\vec{m}}^{\top}\myvec{0&0 \\ 0&1}\brak{\vec{q}+\lambda\vec{m}}-4\myvec{0&1}\brak{\vec{q}+\lambda\vec{m}} =0 \label{eq:eqnII}
\end{align}
Now subtracting \eqref{eq:eqnII} from \eqref{eq:eqnI},
\begin{align}
  \brak{\vec{q}+\lambda\vec{m}}^{\top}\brak{\vec{q}+\lambda\vec{m}}-8\myvec{1&0}\brak{\vec{q}+\lambda\vec{m}}-\brak{\vec{q}+\lambda\vec{m}}^{\top}\myvec{0&0 \\ 0&1}\brak{\vec{q}+\lambda\vec{m}}+4\myvec{0&1}\brak{\vec{q}+\lambda\vec{m}} =0 
\end{align}













\end{document}
